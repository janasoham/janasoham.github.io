%%%%%%%%%%%%%%%%% PREAMBLE %%%%%%%%%%%%%%%%%%%%%%%%%%%%
%Change the font size of your document - 10pt, 12.1pt, etc.
\documentclass[letterpaper,12pt,oneside]{article}
\usepackage[utf8]{inputenc}
\usepackage{setspace}
\usepackage{longtable}
%\graphicspath{ {images/}} %upload your signature to this file
%Change the margins to fit your CV/resume content
\usepackage[left=1in, right=1in, bottom=0.8in, top=0.8in]{geometry}

%Skype information - include your Skype name for a link to add you on Skype
\newcommand*{\Skype}{\href{skype:john.smith?add}{john.smith}} 
\newcommand{\Absender}[1][\normalsize]{\Skype} 

%Changes the page numbers - {arabic}=arabic numerals, {gobble}=no page numbers, {roman}=Roman numerals
\pagenumbering{gobble}


\usepackage{xspace,exscale,relsize}
\usepackage{fancybox,shadow}
\usepackage{graphicx}
\usepackage{color}
% this is for \mathfrak & better working \mathbb
\usepackage{amsthm,amsfonts}
% this is for \gtrsim, \lesssim symbols (redefined as \simleq, \simgeq in my_style.sty)
\usepackage{amssymb}
%\usepackage{url}
\usepackage{authblk}
\usepackage[title]{appendix}

\usepackage{extarrows}

%automatic sorting/compression of citations
% noadjust -- to prevent automatic adding of ~ before \cite{}
\usepackage[noadjust]{cite}

% amsmath is for align, multline env's.
%  -- cmex10 option directs TeX to use rescaled cmex10 for cmex7-9 fonts.
%  	This option is highly recommended by IEEEtran because older systems
%  	did not include Type1 font definitions for cmex7-9 fonts. However, on my
%  	systems it seems that DVIPS is instructed by 
%  	/usr/share/texmf-texlive/fonts/map/dvips/cmex/ttcmex.map:
%  		cmex7 TeX-cmex7 <fmex7.pfb
%		cmex8 TeX-cmex8 <fmex8.pfb
%		cmex9 TeX-cmex9 <fmex9.pfb
%	Thus, I _DO NOT_ need cmex10 option
%\usepackage[cmex10]{amsmath}
\usepackage{amsmath,mathrsfs}
% AMSMath was so ample as to prohibit usage of \over, so we undo it
\makeatletter
\let\over=\@@over \let\overwithdelims=\@@overwithdelims
\let\atop=\@@atop \let\atopwithdelims=\@@atopwithdelims
\let\above=\@@above \let\abovewithdelims=\@@abovewithdelims
\makeatother
% Also, note that the amsmath package sets \interdisplaylinepenalty to 10000
% thus preventing page breaks from occurring within multiline equations. Use:
\interdisplaylinepenalty=10000
% after loading amsmath to restore such page breaks as IEEEtran.cls normally
% does. 





\usepackage{rotating}


%\usepackage{stfloats}
% stfloats.sty was written by Sigitas Tolusis. This package gives LaTeX2e
% the ability to do double column floats at the bottom of the page as well
% as the top. (e.g., "\begin{figure*}[!b]" is not normally possible in
	% LaTeX2e). It also provides a command:
	%\fnbelowfloat
	% to enable the placement of footnotes below bottom floats (the standard
	% LaTeX2e kernel puts them above bottom floats). This is an invasive package
	% which rewrites many portions of the LaTeX2e float routines. It may not work
	% with other packages that modify the LaTeX2e float routines. The latest
	% version and documentation can be obtained at:
	% http://www.ctan.org/tex-archive/macros/latex/contrib/sttools/
	% Documentation is contained in the stfloats.sty comments as well as in the
	% presfull.pdf file. Do not use the stfloats baselinefloat ability as IEEE
	% does not allow \baselineskip to stretch. Authors submitting work to the
	% IEEE should note that IEEE rarely uses double column equations and
	% that authors should try to avoid such use. Do not be tempted to use the
	% cuted.sty or midfloat.sty packages (also by Sigitas Tolusis) as IEEE does
	% not format its papers in such ways.
	
	% correct bad hyphenation here
	%\hyphenation{op-tical net-works semi-conduc-tor}
	
	
	
	
	
	%
	\usepackage{ifpdf}
	
	\usepackage{subfigure}
	\usepackage{psfrag}
	
	\usepackage{prettyref,enumerate}
	%\usepackage{enumitem}
	%\usepackage[showerrors,table,dvipsnames*, svgnames*]{xcolor}
	
	
	\usepackage{tikz}
	\usetikzlibrary{arrows}
	\tikzstyle{int}=[draw, fill=blue!20, minimum size=2em]
	\tikzstyle{dot}=[circle, draw, fill=blue!20, minimum size=2em]
	\tikzstyle{init} = [pin edge={to-,thin,black}]
	
	
	\usepackage[%dvips,
	CJKbookmarks=true,
	bookmarksnumbered=true,
	bookmarksopen=true,
	%						bookmarks=false,
	colorlinks=true,
	citecolor=red,
	linkcolor=blue,
	anchorcolor=red,
	urlcolor=blue,
	pdfauthor={Soham Jana}
	]{hyperref}
	
	\usepackage[all]{xy}
	
	\usepackage{mathtools}
	%\mathtoolsset{showonlyrefs}
	
	\usepackage{ifthen}
	\newboolean{aos}
	\setboolean{aos}{FALSE}
	
	%%%%%%%%%%%%%%%%%%%%%%%%%%%%%%%%%%%%%%%%%%%%%%%%%%%%%%%%%%%%%%%%%%
	%%%%%%%%%%%%%%%%%%%%%%%%%%%%%%%%%%%%%%%%%%%%%%%%%%%%%%%%%%%%%%%%%%
	%%%%                                                          %%%%
	%%%%                      Local defines                       %%%%
	%%%%                                                          %%%%
	%%%%%%%%%%%%%%%%%%%%%%%%%%%%%%%%%%%%%%%%%%%%%%%%%%%%%%%%%%%%%%%%%%
	%%%%%%%%%%%%%%%%%%%%%%%%%%%%%%%%%%%%%%%%%%%%%%%%%%%%%%%%%%%%%%%%%%
	
	\newcommand{\mats}{\ensuremath{\mathcal{S}}}
	\newcommand{\matc}{\ensuremath{\mathcal{C}}}
	\newcommand{\matx}{\ensuremath{\mathcal{X}}}
	\newcommand{\mata}{\ensuremath{\mathcal{A}}}
	\newcommand{\mate}{\ensuremath{\mathcal{E}}}
	\newcommand{\matu}{\ensuremath{\mathcal{U}}}
	\newcommand{\matb}{\ensuremath{\mathcal{B}}}
	\newcommand{\matr}{\ensuremath{\mathcal{R}}}
	\newcommand{\maty}{\ensuremath{\mathcal{Y}}}
	\newcommand{\matn}{\ensuremath{\mathcal{N}}}
	\newcommand{\matf}{\ensuremath{\mathcal{F}}}
	\newcommand{\matg}{\ensuremath{\mathcal{G}}}
	\newcommand{\matp}{\ensuremath{\mathcal{P}}}
	\newcommand{\matd}{\ensuremath{\mathcal{D}}}
	\newcommand{\matK}{\ensuremath{\mathcal{K}}}
	\newcommand{\matH}{\ensuremath{\mathcal{H}}}
	
	\newcommand{\alx}{\textbf{\sf A}}
	\newcommand{\alf}{\textbf{\sf F}}
	\newcommand{\aly}{\textbf{\sf B}}
	\newcommand{\alw}{\textbf{\sf W}}
	
	\newcommand{\mymmh} {\frac{\mathsmaller M \mathsmaller - \mathsmaller 1}{\mathsmaller 2}} %%SV-apr26
	\newcommand{\mymm} {{\mathsmaller M \mathsmaller - \mathsmaller 1 }} %%SV-apr26
	\newcommand{\figref}[1]{Fig.~{\bf #1}}
	\newcommand{\mreals}{\ensuremath{\mathbb{R}}}
	\newcommand{\mconst}{\ensuremath{\mathrm{const}}}
	\newcommand{\supp}{\ensuremath{\mathrm{supp}}}
	
	\ifx\eqref\undefined
	\newcommand{\eqref}[1]{~(\ref{#1})}
	\fi
	\ifx\mod\undefined
	\def\mod{\mathop{\rm mod}}
	\fi
	
	\usepackage{bm}
	\def\vect#1{\bm{#1}}
	\newcommand{\transp}[1]{{#1^{\mbox{\scriptsize\rm T}}}}
	\newcommand{\vecttr}[1]{{\bf #1^{\mbox{\scriptsize\rm T}}}}
	\newcommand{\norm}[1]{{\left\Vert #1 \right\Vert}}
	\newcommand{\Norm}[1]{{\Vert #1 \Vert}}
	
	\def\eng{\fontencoding{OT1}\selectfont}
	\def\yhat{{\hat y}}
	\def\chat{{\hat c}}
	\def\esssup{\mathop{\rm esssup}}
	\def\argmin{\mathop{\rm argmin}}
	\def\argmax{\mathop{\rm argmax}}
	\def\dim{\mathop{\rm dim}}
	\def\wt{\mathop{\rm wt}}
	\def\exp{\mathop{\rm exp}}
	\def\lspan{\mathop{\rm span}}
	\def\tr{\mathop{\rm tr}}
	\def\rank{\mathop{\rm rank}}
	\def\cov{\mathop{\rm cov}}
	\def\MM{{\bf\rm M}}
	\def\EE{\Expect}
	\DeclareMathOperator\sign{\rm sign}
	\def\LVar{\mathbb{L}\,}
	\def\Var{\mathrm{Var}}
	\def\Cov{\mathrm{Cov}}
	\def\DD{{\bf\rm D}}
	\def\wth{\mathop{\rm wt}}
	\def\PP{\mathbb{P}}
	\def\FF{\mathbb{F}}
	\def\QQ{\mathbb{Q}}
	\def\vchat{\vect{\hat c}}
	\def\vctil{\vect{\tilde c}}
	\def\vectx{\vect{x}}
	\def\vdelta{\vect{\delta}}
	\def\vdeltil{\tilde \delta}
	\def\vecty{\vect{y}}
	\def\diag{\mathop{\rm diag}}
	\def\eig{\mathop{\rm eig}}
	\def\follows{\Longrightarrow}
	\def\equiva{\Leftrightarrow}
	\def\equival{\Longleftrightarrow}
	%\def\eqdef{\stackrel{\triangle}{=}}
	\def\eqdef{\triangleq}
	\def\dd{\partial}
	\def\upto{\nearrow}
	\def\downto{\searrow}
	
	\def\capf{\mathsf{C}}
	\def\veef{\mathsf{V}}
	\def\Leb{\mathrm{Leb}}
	\def\simiid{\stackrel{iid}{\sim}}
	
	%\def\co{{\mathrm{co}}}
	
	
	\def\gacap#1{{1\over 2} \log\left(1+{#1}\right)}
	
	
	
	%%%%%%%%%%%%%%%%%%%%%%%%%%% by Soham %%%%%%%%%%%%%%%%%%%%%%%%%
	
	\newcommand{\scr}[1]{\mathscr{#1}}
	\newcommand{\abs}[1]{\left| #1 \right|}
	\newcommand{\trm}{\calM}
	\newcommand{\newrisk}[1]{U_{#1}}
	\newcommand{\mymat}{M}
	
	%%%%%%%%%%%%%%%%%%%%%%%%%%% by Soham %%%%%%%%%%%%%%%%%%%%%%%%%
	
	
	%%%%%%%%%%%%%%%%%%%%%%%%%%%% by Wu %%%%%%%%%%%%%%%%%%%%%%%%%%%%
	
	\newcommand{\M}{M}
	
	\newcommand{\id}{\text{id}}
	
	\newcommand{\zeros}{\mathbf{0}}
	\newcommand{\ones}{\mathbf{1}}
	\newcommand{\stepa}[1]{\overset{\rm (a)}{#1}}
	\newcommand{\stepb}[1]{\overset{\rm (b)}{#1}}
	\newcommand{\stepc}[1]{\overset{\rm (c)}{#1}}
	\newcommand{\stepd}[1]{\overset{\rm (d)}{#1}}
	\newcommand{\stepe}[1]{\overset{\rm (e)}{#1}}
	
	\newcommand{\I}{\text{(I)}}
	\newcommand{\II}{\text{(II)}}
	\newcommand{\III}{\text{(III)}}
	\newcommand{\IV}{\text{(IV)}}
	
	
	%\newcommand{\ones}{\mathbf{1}}
	\newcommand{\bTheta}{\boldsymbol{\Theta}}
	\newcommand{\btheta}{\vect{\theta}}
	\newcommand{\Poi}{\mathrm{Poi}}
	
	\newcommand{\maxcor}{\mathrm{maxcor}}
	\newcommand{\Unif}{\mathrm{Uniform}}
	\newcommand{\hV}{\hat{V}}
	\newcommand{\hW}{\hat{W}}
	\newcommand{\cost}{\sfM}
	\newcommand{\costi}{\cost^{-1}}
	\newcommand{\floor}[1]{{\left\lfloor {#1} \right \rfloor}}
	\newcommand{\ceil}[1]{{\left\lceil {#1} \right \rceil}}
	
	\newcommand{\blambda}{\bar{\lambda}}
	\newcommand{\reals}{\mathbb{R}}
	\newcommand{\naturals}{\mathbb{N}}
	\newcommand{\integers}{\mathbb{Z}}
	\newcommand{\complex}{\mathbb{C}}
	\newcommand{\Expect}{\mathbb{E}}
	\newcommand{\expect}[1]{\mathbb{E}\left[#1\right]}
	\newcommand{\Prob}{\mathbb{P}}
	\newcommand{\prob}[1]{\mathbb{P}\left[#1\right]}
	\newcommand{\pprob}[1]{\mathbb{P}[#1]}
	\newcommand{\intd}{{\rm d}}
	\newcommand{\TV}{{\rm TV}}
	\newcommand{\Aff}{{\rm Aff}}
	\newcommand{\Mu}{M}
	\newcommand{\dbar}{\bar{d}}
	\newcommand{\LC}{{\sf LC}}
	\newcommand{\PW}{{\sf PW}}
	\newcommand{\htheta}{\hat{\theta}}
	\newcommand{\eexp}{{\rm e}}
	\newcommand{\expects}[2]{\mathbb{E}_{#2}\left[ #1 \right]}
	\newcommand{\diff}{{\rm d}}
	\newcommand{\eg}{e.g.\xspace}
	\newcommand{\ie}{i.e.\xspace}
	\newcommand{\iid}{i.i.d.\xspace}
	\newcommand{\ind}{ind.\xspace}
	\newcommand{\fracp}[2]{\frac{\partial #1}{\partial #2}}
	\newcommand{\fracpk}[3]{\frac{\partial^{#3} #1}{\partial #2^{#3}}}
	\newcommand{\fracd}[2]{\frac{\diff #1}{\diff #2}}
	\newcommand{\fracdk}[3]{\frac{\diff^{#3} #1}{\diff #2^{#3}}}
	\newcommand{\renyi}{R\'enyi\xspace}
	\newcommand{\lpnorm}[1]{\left\|{#1} \right\|_{p}}
	\newcommand{\linf}[1]{\left\|{#1} \right\|_{\infty}}
	\newcommand{\lnorm}[2]{\left\|{#1} \right\|_{{#2}}}
	\newcommand{\Lploc}[1]{L^{#1}_{\rm loc}}
	\newcommand{\hellinger}{d_{\rm H}}
	\newcommand{\Fnorm}[1]{\lnorm{#1}{\rm F}}
	%% parenthesis
	\newcommand{\pth}[1]{\left( #1 \right)}
	\newcommand{\qth}[1]{\left[ #1 \right]}
	\newcommand{\sth}[1]{\left\{ #1 \right\}}
	\newcommand{\bpth}[1]{\Bigg( #1 \Bigg)}
	\newcommand{\bqth}[1]{\Bigg[ #1 \Bigg]}
	\newcommand{\bsth}[1]{\Bigg\{ #1 \Bigg\}}
	\newcommand{\xxx}{\textbf{xxx}\xspace}
	\newcommand{\toprob}{{\xrightarrow{\Prob}}}
	\newcommand{\tolp}[1]{{\xrightarrow{L^{#1}}}}
	\newcommand{\toas}{{\xrightarrow{{\rm a.s.}}}}
	\newcommand{\toae}{{\xrightarrow{{\rm a.e.}}}}
	\newcommand{\todistr}{{\xrightarrow{{\rm D}}}}
	\newcommand{\eqdistr}{{\stackrel{\rm D}{=}}}
	\newcommand{\eqlaw}{{\stackrel{\rm law}{=}}}
	\newcommand{\iiddistr}{{\stackrel{\text{\iid}}{\sim}}}
	\newcommand{\inddistr}{{\stackrel{\text{\ind}}{\sim}}}
	\newcommand{\var}{\Var}
	\newcommand\indep{\protect\mathpalette{\protect\independenT}{\perp}}
	\def\independenT#1#2{\mathrel{\rlap{$#1#2$}\mkern2mu{#1#2}}}
	\newcommand{\Bern}{\text{Bern}}
	\newcommand{\Binom}{\text{Binom}}
	\newcommand{\Bino}{\Binom}
	\newcommand{\iprod}[2]{\left \langle #1, #2 \right\rangle}
	\newcommand{\Iprod}[2]{\langle #1, #2 \rangle}
	\newcommand{\indc}[1]{{\mathbf{1}_{\left\{{#1}\right\}}}}
	\newcommand{\Indc}{\mathbf{1}}
	
	\definecolor{myblue}{rgb}{.8, .8, 1}
	\definecolor{mathblue}{rgb}{0.2472, 0.24, 0.6} % mathematica's Color[1, 1--3]
	\definecolor{mathred}{rgb}{0.6, 0.24, 0.442893}
	\definecolor{mathyellow}{rgb}{0.6, 0.547014, 0.24}
	\def\yp#1{\textcolor{mathyellow}{#1}}
	
	
	\newcommand{\red}{\color{red}}
	\newcommand{\blue}{\color{blue}}
	\newcommand{\nb}[1]{{\sf\blue[#1]}}
	\newcommand{\nbr}[1]{{\sf\red[#1]}}
	
	\newcommand{\tmu}{{\tilde{\mu}}}
	\newcommand{\tf}{{\tilde{f}}}
	\newcommand{\tp}{\tilde{p}}
	\newcommand{\tilh}{{\tilde{h}}}
	\newcommand{\tu}{{\tilde{u}}}
	\newcommand{\tx}{{\tilde{x}}}
	\newcommand{\ty}{{\tilde{y}}}
	\newcommand{\tz}{{\tilde{z}}}
	\newcommand{\tA}{{\tilde{A}}}
	\newcommand{\tB}{{\tilde{B}}}
	\newcommand{\tC}{{\tilde{C}}}
	\newcommand{\tD}{{\tilde{D}}}
	\newcommand{\tE}{{\tilde{E}}}
	\newcommand{\tF}{{\tilde{F}}}
	\newcommand{\tG}{{\tilde{G}}}
	\newcommand{\tH}{{\tilde{H}}}
	\newcommand{\tI}{{\tilde{I}}}
	\newcommand{\tJ}{{\tilde{J}}}
	\newcommand{\tK}{{\tilde{K}}}
	\newcommand{\tL}{{\tilde{L}}}
	\newcommand{\tM}{{\tilde{M}}}
	\newcommand{\tN}{{\tilde{N}}}
	\newcommand{\tO}{{\tilde{O}}}
	\newcommand{\tP}{{\tilde{P}}}
	\newcommand{\tQ}{{\tilde{Q}}}
	\newcommand{\tR}{{\tilde{R}}}
	\newcommand{\tS}{{\tilde{S}}}
	\newcommand{\tT}{{\tilde{T}}}
	\newcommand{\tU}{{\tilde{U}}}
	\newcommand{\tV}{{\tilde{V}}}
	\newcommand{\tW}{{\tilde{W}}}
	\newcommand{\tX}{{\tilde{X}}}
	\newcommand{\tY}{{\tilde{Y}}}
	\newcommand{\tZ}{{\tilde{Z}}}
	
	\newcommand{\sfa}{{\mathsf{a}}}
	\newcommand{\sfb}{{\mathsf{b}}}
	\newcommand{\sfc}{{\mathsf{c}}}
	\newcommand{\sfd}{{\mathsf{d}}}
	\newcommand{\sfe}{{\mathsf{e}}}
	\newcommand{\sff}{{\mathsf{f}}}
	\newcommand{\sfg}{{\mathsf{g}}}
	\newcommand{\sfh}{{\mathsf{h}}}
	\newcommand{\sfi}{{\mathsf{i}}}
	\newcommand{\sfj}{{\mathsf{j}}}
	\newcommand{\sfk}{{\mathsf{k}}}
	\newcommand{\sfl}{{\mathsf{l}}}
	\newcommand{\sfm}{{\mathsf{m}}}
	\newcommand{\sfn}{{\mathsf{n}}}
	\newcommand{\sfo}{{\mathsf{o}}}
	\newcommand{\sfp}{{\mathsf{p}}}
	\newcommand{\sfq}{{\mathsf{q}}}
	\newcommand{\sfr}{{\mathsf{r}}}
	\newcommand{\sfs}{{\mathsf{s}}}
	\newcommand{\sft}{{\mathsf{t}}}
	\newcommand{\sfu}{{\mathsf{u}}}
	\newcommand{\sfv}{{\mathsf{v}}}
	\newcommand{\sfw}{{\mathsf{w}}}
	\newcommand{\sfx}{{\mathsf{x}}}
	\newcommand{\sfy}{{\mathsf{y}}}
	\newcommand{\sfz}{{\mathsf{z}}}
	\newcommand{\sfA}{{\mathsf{A}}}
	\newcommand{\sfB}{{\mathsf{B}}}
	\newcommand{\sfC}{{\mathsf{C}}}
	\newcommand{\sfD}{{\mathsf{D}}}
	\newcommand{\sfE}{{\mathsf{E}}}
	\newcommand{\sfF}{{\mathsf{F}}}
	\newcommand{\sfG}{{\mathsf{G}}}
	\newcommand{\sfH}{{\mathsf{H}}}
	\newcommand{\sfI}{{\mathsf{I}}}
	\newcommand{\sfJ}{{\mathsf{J}}}
	\newcommand{\sfK}{{\mathsf{K}}}
	\newcommand{\sfL}{{\mathsf{L}}}
	\newcommand{\sfM}{{\mathsf{M}}}
	\newcommand{\sfN}{{\mathsf{N}}}
	\newcommand{\sfO}{{\mathsf{O}}}
	\newcommand{\sfP}{{\mathsf{P}}}
	\newcommand{\sfQ}{{\mathsf{Q}}}
	\newcommand{\sfR}{{\mathsf{R}}}
	\newcommand{\sfS}{{\mathsf{S}}}
	\newcommand{\sfT}{{\mathsf{T}}}
	\newcommand{\sfU}{{\mathsf{U}}}
	\newcommand{\sfV}{{\mathsf{V}}}
	\newcommand{\sfW}{{\mathsf{W}}}
	\newcommand{\sfX}{{\mathsf{X}}}
	\newcommand{\sfY}{{\mathsf{Y}}}
	\newcommand{\sfZ}{{\mathsf{Z}}}
	
	
	\newcommand{\calA}{{\mathcal{A}}}
	\newcommand{\calB}{{\mathcal{B}}}
	\newcommand{\calC}{{\mathcal{C}}}
	\newcommand{\calD}{{\mathcal{D}}}
	\newcommand{\calE}{{\mathcal{E}}}
	\newcommand{\calF}{{\mathcal{F}}}
	\newcommand{\calG}{{\mathcal{G}}}
	\newcommand{\calH}{{\mathcal{H}}}
	\newcommand{\calI}{{\mathcal{I}}}
	\newcommand{\calJ}{{\mathcal{J}}}
	\newcommand{\calK}{{\mathcal{K}}}
	\newcommand{\calL}{{\mathcal{L}}}
	\newcommand{\calM}{{\mathcal{M}}}
	\newcommand{\calN}{{\mathcal{N}}}
	\newcommand{\calO}{{\mathcal{O}}}
	\newcommand{\calP}{{\mathcal{P}}}
	\newcommand{\calQ}{{\mathcal{Q}}}
	\newcommand{\calR}{{\mathcal{R}}}
	\newcommand{\calS}{{\mathcal{S}}}
	\newcommand{\calT}{{\mathcal{T}}}
	\newcommand{\calU}{{\mathcal{U}}}
	\newcommand{\calV}{{\mathcal{V}}}
	\newcommand{\calW}{{\mathcal{W}}}
	\newcommand{\calX}{{\mathcal{X}}}
	\newcommand{\calY}{{\mathcal{Y}}}
	\newcommand{\calZ}{{\mathcal{Z}}}
	
	\newcommand{\bara}{{\bar{a}}}
	\newcommand{\barb}{{\bar{b}}}
	\newcommand{\barc}{{\bar{c}}}
	\newcommand{\bard}{{\bar{d}}}
	\newcommand{\bare}{{\bar{e}}}
	\newcommand{\barf}{{\bar{f}}}
	\newcommand{\barg}{{\bar{g}}}
	\newcommand{\barh}{{\bar{h}}}
	\newcommand{\bari}{{\bar{i}}}
	\newcommand{\barj}{{\bar{j}}}
	\newcommand{\bark}{{\bar{k}}}
	\newcommand{\barl}{{\bar{l}}}
	\newcommand{\barm}{{\bar{m}}}
	\newcommand{\barn}{{\bar{n}}}
	%\newcommand{\baro}{{\bar{o}}}
	\newcommand{\barp}{{\bar{p}}}
	\newcommand{\barq}{{\bar{q}}}
	\newcommand{\barr}{{\bar{r}}}
	\newcommand{\bars}{{\bar{s}}}
	\newcommand{\bart}{{\bar{t}}}
	\newcommand{\baru}{{\bar{u}}}
	\newcommand{\barv}{{\bar{v}}}
	\newcommand{\barw}{{\bar{w}}}
	\newcommand{\barx}{{\bar{x}}}
	\newcommand{\bary}{{\bar{y}}}
	\newcommand{\barz}{{\bar{z}}}
	\newcommand{\barA}{{\bar{A}}}
	\newcommand{\barB}{{\bar{B}}}
	\newcommand{\barC}{{\bar{C}}}
	\newcommand{\barD}{{\bar{D}}}
	\newcommand{\barE}{{\bar{E}}}
	\newcommand{\barF}{{\bar{F}}}
	\newcommand{\barG}{{\bar{G}}}
	\newcommand{\barH}{{\bar{H}}}
	\newcommand{\barI}{{\bar{I}}}
	\newcommand{\barJ}{{\bar{J}}}
	\newcommand{\barK}{{\bar{K}}}
	\newcommand{\barL}{{\bar{L}}}
	\newcommand{\barM}{{\bar{M}}}
	\newcommand{\barN}{{\bar{N}}}
	\newcommand{\barO}{{\bar{O}}}
	\newcommand{\barP}{{\bar{P}}}
	\newcommand{\barQ}{{\bar{Q}}}
	\newcommand{\barR}{{\bar{R}}}
	\newcommand{\barS}{{\bar{S}}}
	\newcommand{\barT}{{\bar{T}}}
	\newcommand{\barU}{{\bar{U}}}
	\newcommand{\barV}{{\bar{V}}}
	\newcommand{\barW}{{\bar{W}}}
	\newcommand{\barX}{{\bar{X}}}
	\newcommand{\barY}{{\bar{Y}}}
	\newcommand{\barZ}{{\bar{Z}}}
	
	
	
	\newcommand{\hX}{\hat{X}}
	
	\newcommand{\trans}{^{\rm T}}
	\newcommand{\Th}{{^{\rm th}}}
	\newcommand{\diverge}{\to \infty}
	\newcommand{\mmse}{\mathsf{mmse}}
	
	% for prettyref
	\newrefformat{eq}{(\ref{#1})}
	\newrefformat{thm}{Theorem~\ref{#1}}
	\newrefformat{th}{Theorem~\ref{#1}}
	\newrefformat{chap}{Chapter~\ref{#1}}
	\newrefformat{sec}{Section~\ref{#1}}
	\newrefformat{seca}{Section~\ref{#1}}
	\newrefformat{algo}{Algorithm~\ref{#1}}
	\newrefformat{fig}{Fig.~\ref{#1}}
	\newrefformat{tab}{Table~\ref{#1}}
	\newrefformat{rmk}{Remark~\ref{#1}}
	\newrefformat{clm}{Claim~\ref{#1}}
	\newrefformat{def}{Definition~\ref{#1}}
	\newrefformat{cor}{Corollary~\ref{#1}}
	\newrefformat{lmm}{Lemma~\ref{#1}}
	\newrefformat{prop}{Proposition~\ref{#1}}
	\newrefformat{pr}{Proposition~\ref{#1}}
	\newrefformat{app}{Appendix~\ref{#1}}
	\newrefformat{apx}{Appendix~\ref{#1}}
	\newrefformat{ex}{Example~\ref{#1}}
	\newrefformat{exer}{Exercise~\ref{#1}}
	\newrefformat{soln}{Solution~\ref{#1}}
	%%%%%%%%%%%%%%%%%%%%%%%%%%%% by Wu %%%%%%%%%%%%%%%%%%%%%%%%%%%%
	
	\def\unifto{\mathop{{\mskip 3mu plus 2mu minus 1mu%
				\setbox0=\hbox{$\mathchar"3221$}%
				\raise.6ex\copy0\kern-\wd0%
				\lower0.5ex\hbox{$\mathchar"3221$}}\mskip 3mu plus 2mu minus 1mu}}
	
	\def\ds{\displaystyle}
	
	%% These two are defined in amssymb package
	\ifx\lesssim\undefined
	\def\simleq{{{\mskip 3mu plus 2mu minus 1mu%
				\setbox0=\hbox{$\mathchar"013C$}%
				\raise.2ex\copy0\kern-\wd0%
				\lower0.9ex\hbox{$\mathchar"0218$}}\mskip 3mu plus 2mu minus 1mu}}
	\else
	\def\simleq{\lesssim}
	\fi
	
	\ifx\gtrsim\undefined
	\def\simgeq{{{\mskip 3mu plus 2mu minus 1mu%
				\setbox0=\hbox{$\mathchar"013E$}%
				\raise.2ex\copy0\kern-\wd0%
				\lower0.9ex\hbox{$\mathchar"0218$}}\mskip 3mu plus 2mu minus 1mu}}
	\else
	\def\simgeq{\gtrsim}
	\fi
	
	
	% beautiful system of 2 alternatives preceded by {
		\newcommand{\systalt}[4]{\left\{%
			\vcenter{%
				\hbox{%
					\vbox{\hbox{$#1$}\hbox{$#3$}}%
					\quad%
					\vbox{\hbox{$#2$}\hbox{$#4$}}%
				}%
			} \right.%
		}
		\newcommand{\trisystalt}[6]{\left\{%
			\vcenter{%
				\hbox{%
					\vbox{\hbox{$#1$}\hbox{$#3$}\hbox{$#5$}}%
					\quad%
					\vbox{\hbox{$#2$}\hbox{$#4$}\hbox{$#6$}}%
				}%
			} \right.%
		}
		
		\def\tsub{\mathtt{t}}
		
		\def\dperp{\perp\!\!\!\perp}
		
		\def\delchi{\delta_{\chi^2}}
		\def\delTV{\delta_{TV}}
		
		% Macro to show off newstuff to authors
		\long\def\newstuff#1{{\color{mathred}#1}}
		
		\newtheorem{theorem}{Theorem}
		\newtheorem{lemma}[theorem]{Lemma}
		\newtheorem{corollary}[theorem]{Corollary}
		\newtheorem{coro}[theorem]{Corollary}
		\newtheorem{proposition}[theorem]{Proposition}
		\newtheorem{prop}[theorem]{Proposition}
		
		\theoremstyle{definition}
		\newtheorem{definition}{Definition}
		\newtheorem{example}{Example}
		\newtheorem{remark}{Remark}
		
		%%%%%%%%%%%%%%%%%%%%%%%%%%%%%%%%%%%%%%%%%%%%%%%%%%%%%%%%%%%%%%%%
		%%%%%%%%%%%% CONDITIONAL COMPILATION TRICK %%%%%%%%%%%%%%%%%%%%%
		%%%%%%%%%%%% Usage: \ifmapx TEXT \fi 
		%%%%%%%%%%%% The "TEXT" will only appear if *_apx.tex is compiled   
		%%%%%%%%%%%%%%%%%%%%%%%%%%%%%%%%%%%%%%%%%%%%%%%%%%%%%%%%%%%%%%%%
		%
		% Ok, all this pain is to produce the name "xxx_apx" with characters
		% of category 12, since this is what jobname expands to. What I used is the
		% trick that \string\abc produces a sequence of tokens (\,a,b,c) all of category 12, 
		% so I only need to kill the front \_12 which is what mkillslash macro doing.
		%
		%%%%%%%%%%%%
		\newif\ifmapx
		{\catcode`/=0 \catcode`\\=12/gdef/mkillslash\#1{#1}}
		\edef\jobnametmp{\expandafter\string\csname ic_apx\endcsname}
		\edef\jobnameapx{\expandafter\mkillslash\jobnametmp}
		\edef\jobnameexpand{\jobname}
		\ifx\jobnameexpand\jobnameapx
		\mapxtrue
		\else
		\mapxfalse
		\fi
		
		\long\def\apxonly#1{\ifmapx{\color{blue}#1}\fi}
		
		%%%%%%%%%%%%%%%%%%%%%%%%%%%%%%%%%%%%%%%%%%%%%%%%%%%%%%%%%%%%%%%%%%
		%%%%%%%%%%%%%%%%%%%%%%%%%%%%%%%%%%%%%%%%%%%%%%%%%%%%%%%%%%%%%%%%%%
		%%%%                                                          %%%%
		%%%%                Document begins here                      %%%%
		%%%%                                                          %%%%
		%%%%%%%%%%%%%%%%%%%%%%%%%%%%%%%%%%%%%%%%%%%%%%%%%%%%%%%%%%%%%%%%%%
		%%%%%%%%%%%%%%%%%%%%%%%%%%%%%%%%%%%%%%%%%%%%%%%%%%%%%%%%%%%%%%%%%%
		
		\newcommand{\Lip}{\mathop{\mathrm{Lip}}}
		\newcommand{\co}{\mathop{\mathrm{co}}}
		
		\newcommand{\bfH}{\mathbf{H}}
		
		\newcommand{\poly}{\mathsf{poly}}
		\newcommand{\polylog}{\mathsf{polylog}}
		
		\newcommand{\Regret}{\mathsf{Regret}}
		\newcommand{\AvgRegret}{\mathsf{AvgRegret}}
		\newcommand{\AccRegret}{\mathsf{AccRegret}}
		
		\newcommand{\Red}{\mathsf{Red}}
		\newcommand{\Risk}{\mathsf{Risk}}
		
		\newcommand{\SubG}{\mathsf{SubG}}
		
		\renewcommand{\hat}{\widehat}
		\renewcommand{\tilde}{\widetilde}
		
		%%%%%%%%%%%%%%%%% END OF PREAMBLE %%%%%%%%%%%%%%%%%%%%%
		
		\begin{document}
			
			%%%%%%%%%%%%%%%%% NAME OF APPLICANT %%%%%%%%%%%%%%%%%%%
			
			\noindent  \Huge{Soham Jana} \\
			\vspace{1ex} 
			%	\hrule 
			\normalsize
			
			%%%%%%%%%%%%%%%%% CONTACT INFORMATION %%%%%%%%%%%%%%%%%
			% Your email address, website, and Skype name are links to send email, open your website and add you on Skype. 
			%	\noindent {Yale University}
			%	{\hspace{3in}\parbox{\linewidth}{email: 
					%			}}
			\noindent{\begin{minipage}{3.9in}
					240 Hayes-Healy Hall\\
					University of Notre Dame\\
					Notre Dame, IN, USA\\
			\end{minipage}}
			\hfill
			\begin{minipage}{2.7in}
				Updated on: \today\\
				%\today \\
				Website: \href{https://janasoham.github.io/}{https://janasoham.github.io}\\
				Email: {sjana2-at-nd-dot-edu}\\
				Phone: +1 574-631-5503
			\end{minipage}
			\vspace{0.8cm}
			
			%%%%%%%%%%%%%%%%% MAIN BODY %%%%%%%%%%%%%%%%%%%%%%%%%%%
			% The main body is contained in a tabular environment. To move sections onto the next page, simply end the tabular environment and begin a new tabular environment.
			
			
			
			\noindent{\bf Research Interests}
			
			
			\begin{itemize}
				\item[] Theoretical and methodological aspects of high-dimensional statistics, robust estimation, neural networks, causal inference.
			\end{itemize}
			
			\noindent{\bf Education}
			
			\begin{itemize}
				\item[] {\bf PhD. in Statistics and Data Science}\hfill May 2022\\
				Yale University, New Haven, CT, USA\\
				Thesis: Learning non-parametric and high-dimensional distributions\\ 
				via information-theoretic
				methods \\
				Advisor: {Prof. Yihong Wu}  
				
				\item[] {\bf 	Master of Statistics (Hons.)} (First class with distinction) \hfill May 2017\\
				Indian Statistical Institute, Kolkata, West Bengal, India\\
				Specialization: Theoretical Statistics\\
				Dissertation: Characterization of single-integral non-kernel divergences \\
				Advisor: Prof. Ayanendranath Basu
				
				\item[] 
				{\bf Bachelor of Statistics (Hons.)} (First class with Distinction) \hfill May 2015 \\
				{Indian Statistical Institute, Kolkata, West Bengal, India}  
			\end{itemize}
			
			\noindent{\bf Work experiences}
			
			\begin{itemize}
				
				\item[] {\bf University of Notre Dame, Notre Dame, IN, USA}
				
				Assistant Professor, Department of Applied and \hfill August 2024 --  Current\\
				Computational Mathematics and Statistics.
				
				\item[] {\bf Princeton University, Princeton, NJ, USA}
				
				Postdoc,  Department of Operations Research and 
				\hfill June 2022 -- July 2024\\ Financial Engineering \\
				Hosts: Prof. Sanjeev Kulkarni and Prof. Jianqing Fan 
				
				Researcher, The First Republic Bank Research and  \hfill June 2022 -- May 2023\\
				Lifelong Learning Program
				
				Lecturer \hfill Spring 2023 and Fall 2023
				
			\end{itemize}
			
			
			\noindent {\bf Preprints} (``$\ast$": Authors list not in alphabetical order)
			
			\begin{enumerate}
				
				\item Shange Tang, Soham Jana, Jianqing Fan. \href{https://www.arxiv.org/abs/2408.12564}{\bf Factor adjusted spectral clustering for mixture models}. arXiv preprint arXiv:2408.12564 (2024).
				
				
				\item Soham Jana, Jianqing Fan, Sanjeev Kulkarni*. \href{https://arxiv.org/abs/2401.05574}{\bf A general theory for robust clustering via trimmed mean}. arXiv preprint arXiv:2401.05574 (2024).
				
				\item Soham Jana, Kun Yang, and Sanjeev Kulkarni*. \href{https://arxiv.org/abs/2306.09977}{\bf Adversarially robust clustering with optimality guarantees}. arXiv preprint arXiv:2306.09977 (2023).
				
				\item Soham Jana, Yury Polyanskiy, and Yihong Wu. \href{https://arxiv.org/abs/2209.01328}{\bf Optimal empirical Bayes estimation for the Poisson model via minimum-distance methods}. arXiv preprint arXiv:2209.01328 (2022).
				
				
			\end{enumerate}
			
			\noindent {\bf Journal publications} (``$\ast$": Authors list not in alphabetical order)
			
			\begin{enumerate}
				\item Soham Jana, Henry Li, Yutaro Yamada, and Ofir Lindenbaum. {\bf \href{https://www.sciencedirect.com/science/article/abs/pii/S0165168423002670}{\bf Support recovery with Stochastic Gates: theory and application for linear models}}. Elsevier Signal Processing (2023), 213, p.109193.
				
				\item Yanjun Han, Soham Jana and Yihong Wu, \href{https://ieeexplore.ieee.org/abstract/document/10028667}{\bf Optimal Prediction of Markov Chains With and Without Spectral Gap}, in IEEE Transactions on Information Theory, vol. 69, no. 6, pp. 3920-3959, June 2023, doi: 10.1109/TIT.2023.3239508. (\textbf{Extended from the NeurIPS version with analysis of higher-order Markov chains and different loss functions})
				
				\item Soham Jana and Ayanendranath Basu.* \href{https://janasoham.github.io/files/bregman_charac.pdf}{\bf A characterization of all single-integral, non-kernel divergence estimators}. IEEE Transactions on Information Theory 65.12 (2019): 7976-7984.
				
			\end{enumerate}
			
			\noindent {\bf Conference publications} (``$\ast$": Authors list not in alphabetical order)
			
			\begin{enumerate}
				\item Soham Jana, Yury Polyanskiy, Anzo Teh, and Yihong Wu. \href{https://arxiv.org/abs/2307.02070}{\bf Empirical Bayes via ERM and Rademacher complexities: the Poisson model}. In Conference on Learning Theory 2023 Jul 15, PMLR 195:5199-5235.
				
				\item Yanjun Han, Soham Jana, and Yihong Wu. \href{https://arxiv.org/abs/2106.13947}{\bf Optimal prediction of Markov chains with and without spectral gap}. NeurIPS 2021.
				
				\item Soham Jana, Yury Polyanskiy, and Yihong Wu. \href{https://arxiv.org/abs/2005.10561}{\bf Extrapolating the profile of a finite population}. In Conference on Learning Theory 2020 Jul 15 (pp. 2011-2033). PMLR.
			\end{enumerate}
			
			
			%	\noindent\textbf{Practical work}
			%	\begin{itemize}
				%		\item[]  With Mitsuru Igami. \textbf{Geo-spatial exploration of New South Wales gasoline market}
				%	\end{itemize}
			%	
			\noindent \textbf{Invited Talks}
			\begin{itemize}
				\item[] International Indian Statistical Association
				\hfill December 2024\\
				Cochin, Kerala, India
				
				\item[] Joint Statistical Meetings
				\hfill August 2024\\
				Portland, OR, USA 
				
				\item[] University of Notre Dame Statistics Department Seminar
				\hfill February 2024\\
				Notre Dame, IN, USA
				
				\item[] University of Wisconsin-Madison Statistics Department Seminar
				\hfill February 2024\\
				Madison, WI, USA
				
				\item[] University of Texas at Dallas Statistics Department Seminar
				\hfill January 2024\\
				Richardson, TX, USA
				
				\item[] Indian Statistical Institute ISRU Department Seminar \hfill July 2023\\
				Kolkata, West Bengal, India
				
				\item[] Conference on Learning Theory (COLT) \hfill July 2023\\
				Bangalore, Karnataka, India
				
				\item[] Neural Information Processing systems (NeurIPS) \hfill December 2021\\
				Virtual
				
				\item[] Conference on Learning Theory (COLT) \hfill July 2020\\
				Virtual
				
			\end{itemize}
			
			
			\noindent{\bf Teaching}
			\begin{itemize}
				
				\item[] {\bf University of Notre Dame} 
				
				{Introduction to probability} (ACMS 30530) \hfill Fall 2024
				%Princeton University
				
				\item[] {\bf Princeton University} 
				
				{Probability and stochastic systems} (ORF 309/ENG 309/MAT 380) \hfill Spring 2023
				%Princeton University
				
				{Statistical machine learning} (ORF 570)  \hfill Fall 2023\\
				%Princeton University\\
			\end{itemize}
			
%			\noindent{\bf Graduate teaching assistant} (at Yale University)
%			\begin{itemize}
%				\item[] {\bf Stochastic processes} \hfill Spring 2021\\
%				S\&DS 351--551. 
%				Instructor: Prof. Joseph Chang
%				
%				\item[] {\bf Information theory}  
%				\hfill Fall 2020\\
%				S\&DS 364--664.
%				%\hfill Yale University\\
%				Instructor: Prof. Andrew Barron
%				\item[] {\bf Probability theory} \hfill Fall 2019\\
%				S\&DS 241--541. 
%				%\hfill Yale University\\
%				Instructor: Prof. Winston Lin
%				\item[] {\bf Advanced probability} \hfill Spring 2019\\
%				S\&DS 400--600. 
%				%\hfill Yale University\\
%				Instructor: Prof. Sekhar Tatikonda
%				\item[] {\bf Statistical inference} \hfill Fall 2018\\
%				S\&DS 410--610.
%				%\hfill Yale University\\
%				Instructor: Prof. Zhou Fan
%				\item[] {\bf Stochastic Process} \hfill Spring 2018\\
%				S\&DS 251--551.
%				%\hfill Yale University\\
%				Instructor: Prof. Sahand Negahban 
%				
%			\end{itemize}	
			
			\noindent{\bf Honors and awards}
			\begin{itemize}
				\item [] INSPIRE Scholarship, Govt. of India \hfill 2012-2017
				\item [] Indian National Mathematical Olympiad (INMO) merit certificate \hfill 2012\\
				(For being among top 75 in the country)
			\end{itemize}
			
			
			\noindent \textbf{Services}
			\begin{itemize}
				\item[]
				{\bf Paper reviewer}\\
				Annals of Statistics (1)\\
				IEEE Transactions on Information Theory (2)\\
				IEEE International Symposium on Information Theory (1)\\
				Stat - an ISI Journal (1)\\
				Algorithmic Learning Theory
				\item[] 
				{\bf Organizational duties at conferences}
				
				{\it Joint Statistical Meetings} \hfill August 2024\\
				{\it Session chair: New Advances in Nonparametric Hypothesis Testing - Part I}\\
				{\it Session chair: New Developments in Non-Euclidean Statistics}
				
				{\it IEEE Conference on Information Sciences and Systems} \hfill March 2024\\
				{\it  Session chair: Machine learning and statistical inference}
				
				\item[] {\bf Yale S\&DS M.A. admisssion committee} \hfill 2021\\
				Reviewer: one of the committee members
				handling over\\ 150 applications
				and making admission recommendations
				
				\item[] {\bf Yale S\&DS graduate reading group} \hfill 2020\\
				Co-organizer
				Scheduled talks and lead discussion sessions
				
				\item[] {\bf Yale Women in Data Science (WiDS) workshop} \hfill 2020\\
				Served as a mentor for Yale undergrad students participating \\
				in the
				%		\href{https://www.widsconference.org/blog_archive/wids-datathon-2020-workshops-worldwide}
				{WiDS Datathon Challenge 2020}
				
				\item[] {\bf South Asian Graduate and Professional Association} \hfill 2018 -- 2021\\
				{\bf at Yale (SAGA)}\\
				Treasurer, core committee member and cultural committee head\\
				Objective: organizing socio-cultural events to promote diversity and inclusion at Yale
				
				
			\end{itemize}
			
			\vfill
			
			%%%%%%%%%%%%%%%%% REFERENCES %%%%%%%%%%%%%%%%%%%%%%%%%%
			% The reference section has links to your references' websites and email addresses.
			%	
			%	\noindent{\bf References}
			%	\begin{itemize}
				%		\item[]
				%	\begin{minipage}[t]{3.1in}
					%		{\bf Sanjeev Kulkarni}\\
					%		William R. Kenan Jr. Professor\\
					%		Electrical Engineering\\
					%		Princeton University\\
					%		Email: kulkarni@princeton.edu
					%	\end{minipage}
				%	\hspace{0.2in}\begin{minipage}[t]{3.1in}
					%		{\bf Jianqing Fan}\\
					%		Frederick L. Moore '18 Professor of Finance\\
					%		Operations Research and Financial Engineering\\
					%		Princeton University\\
					%		Email: jqfan@princeton.edu
					%	\end{minipage}\\
				%	\item[] 
				%	\begin{minipage}[t]{3.1in}
					%		{\bf Yihong Wu}\\
					%		Professor\\
					%		Statistics and Data Science\\
					%		Yale University\\
					%		Email: yihong.wu@yale.edu
					%	\end{minipage}
				%	\hspace{0.2in}\begin{minipage}[t]{3.1in}
					%		{\bf Yury Polyanskiy}\\
					%		Professor\\
					%		Electrical Engineering and Computer Science\\
					%		Massachusetts Institute of Technology\\
					%		Email: yp@mit.edu
					%	\end{minipage}
				%	\end{itemize}
			%
			
			
			%	
			%	
			%	\clearpage
			%	\setlength\parindent{0cm}
			%	\pagenumbering{gobble} %cover letter should be one page, {gobble}=no page number
			%	
			%	
			%	
			%	\begin{flushright}
				%		\today                           \\
				%		\vspace{1em}                              
				%		Home University            \\
				%		Home Department                  \\
				%		Street Address                       \\
				%		City, State. 12345-67899   \\
				%		Phone: +1 (123) 456-7899         \\
				%		\href{mailto:john.smith@email.com}{john.smith@email.com}  \\ %insert your email address here for a clickable link
				%	\end{flushright}
			%	
			%	
			%	\begin{flushleft}
				%		\textbf{Faculty Search Committee}         \\
				%		Name of University \\
				%		Name of Department                  \\
				%		Address of Department \\
				%		City, State. Zip Code
				%	\end{flushleft}
			%	
			
		\end{document}
		
